\documentclass[DM,authoryear,toc]{lsstdoc}
\input{meta}

% Package imports go here.

% Local commands go here.

%If you want glossaries
%\input{aglossary.tex}
%\makeglossaries

\title{Real-bogus classifier -- status report}

% Optional subtitle
% \setDocSubtitle{A subtitle}

\author{%
Nima Sedaghat
}

\setDocRef{DMTN-272}
\setDocUpstreamLocation{\url{https://github.com/lsst-dm/dmtn-272}}

\date{\vcsDate}

% Optional: name of the document's curator
% \setDocCurator{The Curator of this Document}

\setDocAbstract{%
We report the current status of the real-bogus classifier. The report starts with the design and technical aspects and continues to show quantitative and qualitative evaluations.
}

% Change history defined here.
% Order: oldest first.
% Fields: VERSION, DATE, DESCRIPTION, OWNER NAME.
% See LPM-51 for version number policy.
\setDocChangeRecord{%
  \addtohist{1}{YYYY-MM-DD}{Unreleased.}{Nima Sedaghat}
}


\begin{document}

% Create the title page.
\maketitle
% Frequently for a technote we do not want a title page  uncomment this to remove the title page and changelog.
% use \mkshorttitle to remove the extra pages

% ADD CONTENT HERE
% You can also use the \input command to include several content files.

\section{Architecture}
So far, we have run the experiments with three different architectures:

\begin{itemize}
\item{An off-the-shelf architecture based on ResNet50.}
\item{An off-the-shelf architecture based on VGG6.}
\item{A custom architecture based on the encoder part of TransiNet, internally referred to as rbTN}.
\end{itemize}

The former two (ResNet50 and VGG6) have been more extensively tested.

\section{Data}
We used the version of the DC2 dataset available as a butler repository, as the main source of our data.



\appendix
% Include all the relevant bib files.
% https://lsst-texmf.lsst.io/lsstdoc.html#bibliographies
\section{References} \label{sec:bib}
\renewcommand{\refname}{} % Suppress default Bibliography section
\bibliography{local,lsst,lsst-dm,refs_ads,refs,books}

% Make sure lsst-texmf/bin/generateAcronyms.py is in your path
\section{Acronyms} \label{sec:acronyms}
\addtocounter{table}{-1}
\begin{longtable}{p{0.145\textwidth}p{0.8\textwidth}}\hline
\textbf{Acronym} & \textbf{Description}  \\\hline

CCD & Charge-Coupled Device \\\hline
DC2 & Data Challenge 2 (DESC) \\\hline
DM & Data Management \\\hline
DMTN & DM Technical Note \\\hline
TBA & To Be Announced \\\hline
\end{longtable}

% If you want glossary uncomment below -- comment out the two lines above
%\printglossaries





\end{document}
